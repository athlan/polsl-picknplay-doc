\documentclass[11pt,a4paper]{article}
\usepackage{polski}
\usepackage[utf8]{inputenc}
\usepackage[titletoc]{appendix}
% by użyć polskich znaków w systemach Linux 
% używamy kodowania "latin2" lub "utf8", dla Windows "cp1250"

% ustawienia obrazow %
\usepackage{graphicx}
\graphicspath{ {./assets/images/} }
\DeclareGraphicsExtensions{.pdf,.png,.jpg}

% listing kodu
\usepackage{listings}
\usepackage{color}

\definecolor{lightgray}{rgb}{.9,.9,.9}
\definecolor{darkgray}{rgb}{.4,.4,.4}
\definecolor{purple}{rgb}{0.65, 0.12, 0.82}

\lstdefinelanguage{JavaScript}{
  keywords={typeof, new, true, false, catch, function, return, null, catch, switch, var, if, in, while, do, else, case, break},
  keywordstyle=\color{blue}\bfseries,
  ndkeywords={class, export, boolean, throw, implements, import, this},
  ndkeywordstyle=\color{darkgray}\bfseries,
  identifierstyle=\color{black},
  sensitive=false,
  comment=[l]{//},
  morecomment=[s]{/*}{*/},
  commentstyle=\color{purple}\ttfamily,
  stringstyle=\color{red}\ttfamily,
  morestring=[b]',
  morestring=[b]"
}

\lstset{
   language=JavaScript,
   extendedchars=true,
   basicstyle=\ttfamily,
   showstringspaces=false,
   showspaces=false,
   numbers=left,
   numberstyle=\footnotesize,
   numbersep=9pt,
   tabsize=4,
   breaklines=true,
   showtabs=false,
   captionpos=b
}

% document settings

\title{Zdalne sterowanie ze smartfona grą marketingową na interaktywnym bilboardzie LED} 
\author{
  Pelczar Piotr\\
  \and
  Badyla Kamil\\
}
\date{} 

\begin{document}
\maketitle 

\tableofcontents

% pierwsza sekcja
\section{Tekst}label{sectekst}
LaTeX ułatwia autorowi tekstu zarządzanie 
numerowaniem sekcji, wypunktowaniami oraz odwołaniami 
do tabel, rysunków i~innych elementów. W~łatwy sposób 
możemy się odwołać do wzoru \ref{eqnwzor1}.

% druga sekcja
\section{Matematyka}label{secmatematyka}
Poniższy wzór prezentuje możliwości LaTeX w~zakresie 
składu formuł matematycznych. Wzory są numerowane 
automatycznie, podobnie jak inne elementy o~których 
mowa w~sekcji~\ref{sectekst}.
\begin{equation}
E = mc^2,
\label{eqnwzor1}
\end{equation}
gdzie
\begin{equation}
m = frac{m_0}{sqrt{1-frac{v^2}{c^2}}}.
\end{equation}

\section{Cel}
Celem realizacji naszego projektu jest stworzenie platformy pozwalającej uruchamiać aplikację na wielu rozproszonych monitorach sterowanych przez urządzenia wyposażone w przeglądarkę internetową, oraz bezprzewodową kartę sieciową.  
Istotną zaletą systemu jest całkowita niezależność systemu od marki czy modelu ekranu. Bowiem istniejące rozwiązania na rynku są powiązane z jednym systemem operacyjnym i jedną marką.

\newpage
\section{Motywacja}

TODO:
Dlaczego projekt powstaje? Uzasadnienie.


\section{Narzędzia}
\label{cha:Narzędzia}

\subsection{GIT}
\label{sub:GIT}
Rozproszony system kontroli wersji. System ten pozwolił nam na bezproblemową synchronizacje naszej pracy między stacjami roboczymi. Dzięki temu, że jest on rozproszony nie jest konieczne wyznaczanie głównego serwera przez, który synchronizujemy naszą pracę, ale każdy z klientów może pełnić rolę serwera do którego wysyłamy zmiany plików objętych systemem kontroli wersji.


\subsection{Vim}
\label{sub:Vim}
Prawdopodobnie najpopularniejszy konsolowy edytor tekstu. Vim jest klonem edytora vi, z wieloma nowymi funkcjami. Przede wszystkim jest to modalny edytor tesktu tzn. że posiada więcej niż jeden tryb pracy (INPUT, COMMAND, VISUAL). Dzięki swojej prostocie, szybkości, oraz mnogości wtyczek ten konsolowy edytor jest wciąż bardzo popularny i używany w zastępstwie wielu zintegrowanych środowisk programistycznych jak np. Eclipse czy VisualStudio.  



\section{Architektura systemu}

\subsection{Pilot do sterowania zdalnym monitorem}

TODO:

\subsection{Gra wyświetlana na zdalnym monitorze}

TODO:


\subsection{Klient webowy}
\label{subsub:impl-webclient}

Klient webowy został zaimplementowany w obu aplikacjach (pilota do sterowania zdalnym monitorem oraz grze PONG) jako statyczna strona internetowa, której kod JavaScript przechowany jest w Web Storage (podsekcja \ref{subsub:webstorage}) za pomocą basket.js (podsekcja \ref{subsub:tools-basketjs}). Komunikacja z serwerem została zrealizowana przy użyciu biblioteki Socket.io (podsekcja \ref{subsub:socketio}), odbywa się ona za pomocą protokołu Web Sockets (podsekcja \ref{subsub:websockets}) lub, w przypadku braku obsługi standardu przez przeglądarkę internetową, z jedną ze wspieranych metod przez bibliotekę Socket.io z modelu Comet (podsekcja \ref{sub:communication-methods}). W aplikacjach wykrywane są dotknięcia ekranu urządzenia mobilnego (sekcja \ref{sub:touch-detection}) przy użyciu biblioteki jQuery Mobile (podsekcja \ref{subsub:tool-jquery-mobile}).

\subsubsection{Reprezentacja kursora na płaszczyźnie}
\label{subsub:cursor-representation}

Kursor jest reprezentowany za pomocą punktu \(C(x_{c}, y_{c})\) o nieujemnych, całkowitych współrzędnych \emph{cursor coordinate} oznaczających jego pozycję od punktu \(P(0, 0)\) umieszczonego w lewym, górnym rogu ekranu, a wektor jest skierowany ku punktowi w prawym dolnym rogu ekranu.

\begin{figure}[h!]
  \centering
    \includegraphics{wyznaczanie-pozycji-kursora}
  \caption{Sposób wyznaczania pozycji kursora myszy.}
\end{figure}

Uwzględniając różne proporcje \(p = \frac{szerokosc}{wysokosc}\) ekranu, na którym pojawia się zdalnie sterowany kursor oraz proporcje powierzchni ekranu urządzenia mobilnego, w implementacji pilota użyta została reprezentacja procentowa punktu na ekranie.

Pozycja kursora reprezentowana jest przez punkt \(P(x, y)\) na dwuwymiarowej płaszczyźnie w zakresie \( x\in \langle0, 1\rangle \), gdzie \(x = \frac{x_{c}}{w_{x}}\), \(y = \frac{x_{c}}{w_{y}}\), a \(w_{x}\) oraz \(w_{y}\) oznaczają kolejno szerokość i wysokość ekranu.

Wyznaczanie pozycji uwzględnia również zmianę orientacji (poziomej lub pionowej), szerokości i wysokości rzutni okna przeglądarki wyświetlanej na urządzeniu mobilnym.

\lstset{language=JavaScript}
\begin{lstlisting}
this.run = function() {
	
	var this = that
	this.windowX = 0
	this.windowY = 0
	
	$(document).on('vmousemove', function(e) {
		e.preventDefault(); // prevent scroll
		
		var x = e.pageX / that.windowX
		var y = e.pageY / that.windowY
		
		console.log(x, y) // wypisz punkty
	})
	
	var indicateWindowSize = function() {
		that.windowX = $(window).width()
		that.windowY = $(window).height()
	}
	
	// register window size changes listeners
	$(window).on('resize orientationchange', function() {
		indicateWindowSize()
	})
	indicateWindowSize()
}
\end{lstlisting}

\subsubsection{Pilot do sterowania zdalnym monitorem}

Logika aplikacji pilota jest trywialna. Przechwytuje punty zestyku użytkownika z urządzeniem mobilnym, przekształca reprezentację współrzędnych (podsekcja \ref{subsub:cursor-representation}) i wysyła do je do serwera w postaci współrzędnych x, y za pomocą metody \lstinline{emit()} biblioteki socket.io (podsekcja \ref{subsub:socketio}) wiadomością o temacie \emph{mouseMoveToPercent}:

\lstset{language=JavaScript}
\begin{lstlisting}
this.run = function() {
	// init socket connection
	that.socket = socketConnect(that.socket, that.serverAddress);
	
	that.socket.emit('connectToRemote', {
		host: that.serverRemoteHost,
		port: that.serverRemotePort
	})
	
	// register listeners
	$(window).on('resize orientationchange', function() {
		indicateWindowSize()
	})
	indicateWindowSize()
	
	$(document).on('vmousemove', function(e) {
		e.preventDefault(); // prevent scroll
		
		var x = e.pageX / that.windowX
		var y = e.pageY / that.windowY
		
		if(that.x == null || that.y == null || Math.abs(that.x - x) > that.minDelta || Math.abs(that.y - y) > that.minDelta) {
			that.x = x
			that.y = y
			
			var data = {
				x: x,
				y: y
			}
			
			that.socket.emit('mouseMoveToPercent', data)
		}
	})
}
\end{lstlisting}

Aplikacja posiada również zapisy konfiguracji (nazwa hosta oraz port) połączenia do serwera programu generującego zdarzenia urządzeń wejścia (podsekcja \ref{sub:impl-displayclient-events-dispatcher}), do którego ma być przekazywana pozycja, którą wysyła wiadomością o temacie \emph{connectToRemote}. Przykładowa konfiguracja\footnote{Pełen opis instalacji w podsekcji \ref{subsub:setup-server-nodejs}}:

\lstset{language=HTML}
\begin{lstlisting}
<body data-server="192.168.1.127:8080" data-server-remote-host="192.168.1.127" data-server-remote-port="8081">
</body>
\end{lstlisting}

Interfejs wyjściowy aplikacji obejmuje dwa komunikaty: \emph{mouseMoveToPercent} i \emph{connectToRemote}.

\begin{description}
	\item[mouseMoveToPercent] \hfill \\
	Komunikat wysyłany do serwera, informujący o pozycji kursora.
	\begin{enumerate}
		\item x (float)
		\item y (float)
	\end{enumerate}
\end{description}

\begin{description}
	\item[connectToRemote] \hfill \\
	Komunikat wysyłany do serwera, informujący o połączeniu do serwera programu generującego zdarzenia urządzeń wejścia.
	\begin{enumerate}
		\item host (string)
		\item port (float)
	\end{enumerate}
\end{description}

Interfejs wejściowy aplikacji:

Brak.

\subsubsection{Pilot gry PONG wyświetlanej na zdalnym monitorze}

Pilot do sterowania grą PONG wyświetlaną na zdalnym monitorze umożliwia zgłoszenie chęci uczestnictwa w grze oraz sterowanie paletką. Na ekranie startowym użytkownik widzi przycisk, dzięki któremu może zgłosić chęć uczestnictwa w grze, następnie pilot wyświetla komunikaty, aż wreszcie umożliwia sterowanie paletką. Aplikacja jest napisana w jQuery oraz ostylowana arkuszem napisanym w CSS 3.

Interfejs wyjściowy aplikacji obejmuje jeden komunikat: \emph{paddleMove}.

\begin{description}
	\item[paddleMove] \hfill \\
	Komunikat informujący serwer o zmianie położenia punktu zestyku użytkownika z ekranem:
	\begin{enumerate}
		\item positionX (float)
		\item positionY (float)
	\end{enumerate}
\end{description}

Interfejs wejściowy aplikacji (komunikaty wysyłane przez serwer do klienta) obejmuje cztery komunikaty: \emph{signalCurrentServerState}, \emph{signalWaitingForPartner}, \emph{welcome} i \emph{gameOver}.

\begin{description}
	\item[signalCurrentServerState] \hfill \\
	Komunikat informujący o stanie gry oraz kolejki. Odbierany jest za każdym razem, gdy zmienia się stan gry: ilość użytkowników oczekujących na grę, włączenie gry. Na podstawie otrzymanych danych wyświetla się komunikat, który w kolejce jest użytkownik.
	\begin{enumerate}
		\item count (int) liczba użytkowników oczekujących w kolejce do gry
		\item isGame (boolean) informacja, czy jest rozgrywana gra, czy nie
	\end{enumerate}
\end{description}

\begin{description}
	\item[signalWaitingForPartner] \hfill \\
	Komunikat oznaczający, że klient został przyjęty do kolejki oczekujących graczy i oczekuje na partnera do gry.
\end{description}

\begin{description}
	\item[welcome] \hfill \\
	Komunikat oznaczający, że klient rozpoczyna grę.
\end{description}

\begin{description}
	\item[gameOver] \hfill \\
	Komunikat wysyłany przez serwer informujący o zakończeniu gry. Na podstawie otrzymanych danych wyświetla się informacja, czy klient przegrał, czy wygrał.
	\begin{enumerate}
		\item score (int) wynik klienta
		\item scoreEnemy (int) wynik przeciwnika
	\end{enumerate}
\end{description}

\subsection{Serwer}

Serwer został zaimplementowany w obu przypadkach na platformie Node.js (podsekcja \ref{sub:tool-server-nodejs}), która umożliwia asynchroniczne wykonywanie się kodu w pojedynczym wątku (podsekcja \ref{subsub:asyncprogramming} oraz \ref{sub:tool-server-nodejs}).

\subsubsection{Serwer sterowania zdalnym monitorem}

Serwer sterowania zdalnym ekranem przyjmuje każde połączenie w protokole Web Sockets i przekazuje dane do serwera generującego zdarzenia urządzeń wejścia (podsekcja \ref{sub:impl-displayclient-events-dispatcher}) łącząc się z nim uprzednio poprzez standardowe gniazdo. Każdy podłączony klient wysyła również konfigurację połączenia z serwerem generującym zdarzenia urządzeń wejścia.

\par \hfill

\noindent
Interfejs wyjściowy dla serwera:

\par \hfill

\noindent
Dane na temat pozycji kursora wysyłane są w formacie:

\par \hfill

\noindent
x:\emph{POZYCJA X}:y:\emph{POZYCJA Y};

\par \hfill

\noindent
gdzie \emph{POZYCJA X} i \emph{POZYCJA Y} to wartości z przedziału \(\langle0, 1\rangle\) (podsekcja \ref{subsub:cursor-representation}).

\par \hfill

Interfejs wejściowy dla serwera:

\begin{description}
	\item[mouseMoveToPercent] \hfill \\
	Komunikat wysyłany przez pilota, informujący o pozycji kursora.
	\begin{enumerate}
		\item x (float)
		\item y (float)
	\end{enumerate}
\end{description}

\begin{description}
	\item[connectToRemote] \hfill \\
	Komunikat wysyłany przez pilota, informujący o połączeniu do serwera programu generującego zdarzenia urządzeń wejścia. Po jego przyjęciu serwer podejmuje próbę nawiązania połączenia z ów programem.
	\begin{enumerate}
		\item host (string)
		\item port (float)
	\end{enumerate}
\end{description}


\subsubsection{Serwer gry PONG wyświetlanej na zdalnym monitorze}

Serwer gry PONG przyjmuje każde połączenie w protokole Web Sockets oraz dodaje je do kolejki. Zaraz po dodaniu podejmowana jest próba uruchomienia gry dla \(n\) graczy (domyślnie dwóch). Przy każdej zmianie stanu gry (podłączanie i rozłączanie się użytkowników, uruchomienie i zakończenie gry) wysyłana jest informacja do wszystkich o jej zmianie.

Istotnym jest fakt, że serwer jest napisany w taki sposób, aby móc sterować dowolnym programem reprezentującym grę, która spełnia interfejs:

\lstset{language=JavaScript}
\begin{lstlisting}
var gameserver = new gameServer.gameserver(app, server, io);
var clientDispatcher = new clientDispatcher.clientDispatcher(app, server, io, gameserver);
\end{lstlisting}

Pierwszy program to obsługa gry, a drugi to strategia rozgrywki, czyli w jaki sposób gracze mają zostać włączanie do gry. W przypadku gry PONG gracze ustawiani są w kolejce do gry i wybierani sąsiednie parami.

Interfejs wyjściowy dla serwera:

\begin{description}
	\item[signalCurrentServerState] \hfill \\
	Komunikat wysyłany w trybie broadcast do wszystkich użytkowników, informujący o stanie gry oraz kolejki. Wysyłany jest za każdym razem, gdy zmienia się stan gry: ilość użytkowników oczekujących na grę, włączenie gry.
	\begin{enumerate}
		\item count (int) ilość użytkowników oczekujących w kolejce do gry
		\item isGame (boolean) informacja, czy jest rozgrywana gra, czy nie
	\end{enumerate}
\end{description}

\begin{description}
	\item[signalWaitingForPartner] \hfill \\
	Komunikat wysyłany do pierwszego zakolejkowanego użytkownika, oznaczający, że klient został przyjęty do kolejki oczekujących graczy i oczekuje na partnera do gry.
\end{description}

\begin{description}
	\item[welcome] \hfill \\
	Komunikat wysyłany do wybranych użytkowników biorących udział w grze, oznaczający, że klient rozpoczyna grę.
\end{description}

\begin{description}
	\item[gameOver] \hfill \\
	Komunikat wysyłany do wybranych użytkowników biorących udział w grze, informujący o zakończeniu gry.
	\begin{enumerate}
		\item score (int) wynik klienta
		\item scoreEnermy (int) wynik przeciwnika
	\end{enumerate}
\end{description}

Interfejs wejściowy dla serwera:

\begin{description}
	\item[paddleMove] \hfill \\
	Komunikat informujący serwer o zmianie położenia punktu zestyku użytkownika z ekranem:
	\begin{enumerate}
		\item positionX (float)
		\item positionY (float)
	\end{enumerate}
\end{description}

Z jednej strony serwer prowadzi komunikację z pilotami klientów, z drugiej natomiast z aplikacją wyświetlającą grę (podsekcja \ref{sub:impl-displayclient-game}). I tu wymieniane są komunikaty, kolejno:

Interfejs wyjściowy dla serwera:

\begin{description}
	\item[paddleMove] \hfill \\
	Komunikat informujący grę o zmianie położenia paletki gracza.
	\begin{enumerate}
		\item positionX (float)
		\item positionY (float)
		\item paddleId (string) identyfikator paletki (\emph{l} - lewa, \emph{r} - prawa)
	\end{enumerate}
\end{description}

\begin{description}
	\item[signalGameSart] \hfill \\
	Komunikat informujący grę o konieczności jej zakończenia (z przyczyny np. ucieczki jednego z gracza tj. rozłączenia się).
\end{description}

\begin{description}
	\item[signalGameStop] \hfill \\
	Komunikat informujący grę o konieczności jej inicjalizacji oraz rozpoczęcia.
\end{description}

Interfejs wejściowy dla serwera:

\begin{description}
	\item[signalGameStop] \hfill \\
	Komunikat klienta gry o jej zakończeniu wraz z danymi pozwalającymi ustalić wynik. Wynik jest wysyłany do klientów.
	\begin{enumerate}
		\item scoreL (int) wynik gracza po lewej stornie
		\item scoreR (int) wynik gracza po prawej stronie
	\end{enumerate}
\end{description}

\subsection{Program generujący zdarzenia urządzeń wejścia}
\label{sub:impl-displayclient-events-dispatcher}

\subsubsection{Linux}
Aplikacja dla systemu Linux składa się z następujących komponentów: 
\begin{description}
	\item[serwer] \hfill \\
Kod źródłowy serwera znajduje się w załączniku ~\ref{app:server_c}, jest on wspólny dla wszystkich platform unixowych.		
	\item[generator zdarzeń wejścia] \hfill \\ 
		Program generujący zdarzenia urządzeń wejścia na podstawie danych otrzymanych z serwera. W platformie Linux wykorzystano moduł jądra \lstinline{uinput}. Uinput to moduł pozwalający na generowanie zdarzeń symulujących urządzenia wyjścia. Moduł ten tworzy specjalny plik urządzenia w katalogu \lstinline{/dev}. Plik ten jest wirtualnym interfejsem, co znaczy, że nie jest połączony z żadnym fizycznym urządzeniem.

		Jednym z zadań generatora jest ustalenie rozdzielczości ekranu, na którym wyświetlany jest obraz. Informacja ta jest niezbędna, gdyż koordynaty wysyłane przez serwer są podawane absolutnie, oraz z przedziału od 0 do 1 (ze względu na to, że informacja ta nie może być uzależniona od rozdzielczości monitora). Na platformie Linux rozdzielczość uzyskiwana jest za pomocą następującego polecenia:
\begin{lstlisting}[language=bash]
"xdpyinfo -display :0 | sed 's/^ *dimensions: *\\([0-9x]*\\).*/\\1/;t;d'";
\end{lstlisting}
		Program xdpyinfo jest to aplikacja służąca do wyświetlania informacji o serwerze X, z kolei za pomocą wyrażenia regularnego i programu sed wyłuskano tylko niezbędne informacje dotyczące rozdzielczości ekranu.

		Istotnym elementem w pisaniu programu generującego zdarzenia urządzeń wejścia jest odbiór danych z serwera, opisanego w załączniku ~\ref{app:server_c}. Program ten wysyła dane do deskryptora pliku o nazwie \lstinline{STDIN_FILENO}. Nie jest to bynajmniej to samo, co wskaźnik do pliku o nazwie \lstinline{stdin}, przez co nie jest możliwe użycie funkcji wysokiego poziomu, takiej jak na przykład: \lstinline{scanf}, do wczytania danych. Niezbędne jest użycie funkcji \lstinline{read}, a następnie użycie funkcji \lstinline{sscanf} w celu sformatowaniu danych, co pokazuje poniższy fragment kodu:

\begin{lstlisting}[language=c]
read(STDIN_FILENO, buffer, sizeof(buffer));
int result = sscanf(buffer, "x:%lf:y:%lf;", &xTmp, &yTmp);
int x = xTmp * dim[0];
int y = yTmp * dim[1];
\end{lstlisting}

Wczytane koordynaty należy wysłać jako parametry zdarzeń do modułu \lstinline{uinput}. 

\begin{lstlisting}[language=c]
#include <linux/input.h>
#include <linux/uinput.h>
int fd = open("/dev/input/uinput", O_WRONLY | O_NONBLOCK);
ret = ioctl(fd, UI_SET_EVBIT, EV_ABS);

...

struct input_event ev, synEv;


...
memset(&ev, 0, sizeof(ev));
ev.type = EV_ABS;
ev.code = ABS_X;
ev.value = x;
if(write(fd, &ev, sizeof(struct input_event)) < 0)
die("error: write");  

memset(&synEv, 0, sizeof(struct input_event));
synEv.type = EV_SYN;
synEv.code = 0;
synEv.value = 0;
if(write(fd, &synEv, sizeof(struct input_event)) < 0)
die("error: write");

...
//sending y coordinate


\end{lstlisting}
		Na początku należy włączyć niezbędne nagłówki (linie 1-2), następnie otworzyć plik urządzenia \lstinline{uinput}, do którego będą wysyłane zdarzenia urządzeń wejścia. W linii 4 powyższego listingu włączono możliwość wysyłania zdarzeń ruchów myszką, podając koordynaty absolutne. Struktury zadeklarowane w linii 8 zawierają 3 pola: 
\begin{itemize}
\item \lstinline{type} - typ zdarzenia (\lstinline{EV_KEY}, \lstinline{EV_ABS}, \lstinline{EV_REL})		
\item \lstinline{code} - dla zdarzeń klawiatury jest to kod klawisza, natomiast dla zdarzeń myszy jest to oś.
\item \lstinline{value} - dla zdarzeń klawiatury 1 oznacza przyciśnięcie klawisza, a 0 puszczenie, dla zdarzeń myszy jest to z kolei wartość reprezentująca koordynatę.
\end{itemize}
		Po wypełnieniu struktury odpowiednimi wartościami należy wysłać je do wcześniej otwartego pliku (\lstinline{fd}), a następnie wypełnić i wysłać strukturę synchronizującą (\lstinline{evSyn}), której zadaniem jest zatwierdzenie, oraz realizacja wszystkich dotychczasowo wysłanych zdarzeń. 
	\item[serwer X] \hfill \\
	Ta część systemu jest odpowiedzialna za wyświetlanie klientów X.
		Moduł ten opisany jest bardziej szczegółowo w załączniku ~\ref{app:X Window System}.


\end{description}

\subsubsection{Mac OS X}

W niniejszym podrozdziale nie zostanie omówiony proces instalacji systemu Mac OS X na żadnym z jednoukładowych komputerów, gdyż jest to niemożliwe ze względu na to, że system Mac OS X jest dedykowanym oprogramowaniem dla komputerów firmy Apple. 

Serwer odbierający zdarzenia jest wspólny dla wszystkich systemów unixowych i został opisany w załączniku ~\ref{app:server_c}.
\\
Kod generatora zdarzeń dla systemu Mac OS X jest nieco prostszy aniżeli dla systemu Linux. Bazuje on na zdarzeniach z biblioteki CoreGraphics. Tak jak i dla systemu Linux, istotnym elementem jest odczytanie rozdzielczości ekranu. Jest to wykonane za pomocą polecenia:
\begin{lstlisting}
system_profiler SPDisplaysDataType | grep Resolution | sed 's/[^0-9 ]*//g' | sed -e 's/^[ \t]*//' | sed 's/  */x/g'
\end{lstlisting}
Format wyniku powyższego polecenia jest dokładnie taki sam, jak dla instrukcji analogicznej w systemie Linux.
\\
Kod programu odpowiadający za wygenerowanie zdarzenia na podstawie koordynat znajduje się na poniższym listingu:
\begin{lstlisting}
	CGEventRef move1 = CGEventCreateMouseEvent(
	NULL, kCGEventMouseMoved,
	CGPointMake(x, y),
	kCGMouseButtonLeft
);

CGEventPost(kCGHIDEventTap, move1);

CFRelease(move1);
\end{lstlisting}

Funkcja \lstinline{CGEventCreateMouseEvent} tworzy nowe zdarzenie, polegające na przesunięciu kursora myszy do zadanej lokacji.

\subsection{Klient wyświetlający grę PONG na zdalnym monitorze}
\label{sub:impl-displayclient-game}

Klient gry został napisany w JavaScript oraz CSS 3 wykorzystując jego mechanizmy. Zawiera w sobie logikę gry i wysyła komunikaty do serwera sterującego. Wyświetla napis z zaproszeniem do dołączenia do gry bądź (w trakcie jej trwania) paletki wraz z piłką. Klient jest uruchamiany w przeglądarce internetowej wyświetlanej na zdalnym monitorze.

Interfejs wyjściowy dla gry: 

\begin{description}
	\item[signalGameStop] \hfill \\
	Komunikat wysyłany do serwera gry, informujący o jej zakończeniu wraz z danymi pozwalającymi ustalić wynik.
	\begin{enumerate}
		\item scoreL (int) wynik gracza po lewej stornie
		\item scoreR (int) wynik gracza po prawej stronie
	\end{enumerate}
\end{description}

Interfejs wejściowy dla gry:

\begin{description}
	\item[paddleMove] \hfill \\
	Komunikat wysyłany przez serwer gry, informujący o zmianie położenia paletki gracza.
	\begin{enumerate}
		\item positionX (float)
		\item positionY (float)
		\item paddleId (string) identyfikator paletki (\emph{l} - lewa, \emph{r} - prawa)
	\end{enumerate}
\end{description}

\begin{description}
	\item[signalGameSart] \hfill \\
	Komunikat wysyłany przez serwer gry, informujący grę o konieczności jej zakończenia (z przyczyny np. ucieczki jednego z gracza tj. rozłączenia się).
\end{description}

\begin{description}
	\item[signalGameStop] \hfill \\
	Komunikat wysyłany przez serwer gry, informujący grę o konieczności jej inicjalizacji oraz rozpoczęcia.
\end{description}

\section{Serwer zarządzania infrastrukturą}

\subsection{Język Scala}

\subsection{Framework Play}

\subsection{Zdalne sterowanie programami}



\section{Specyfikacja zewnętrzna}

%\section{Przypadki użycia}



\section{Bibliografia, spis tabel i rysunków}

\begin{thebibliography}{1}
  
  %% common bibliography here
  
  \bibitem{message-oriented-middleware} Edward Curry, \emph{Message-Oriented Middleware}, National University of Ireland, Galway, Ireland

  \bibitem{programming-paradigms} Seema Kedar, \emph{Programming Paradigms And Methodology}, Technical Publications, 2008

  \bibitem{programming-async} \emph{Introduction to Asynchronous Programming}, Brown University \url{http://cs.brown.edu/courses/cs196-5/f12/handouts/async.pdf} [Data uzyskania dostępu: 29 grudzień 2013]

  \bibitem{programming-async-sockets} Brian Hall, \emph{Beej's Guide to Network Programming Using Internet Sockets}, 3 czerwca 2012 \url{http://beej.us/guide/bgnet/output/html/multipage/advanced.html} [Data uzyskania dostępu: 29 grudzień 2013]

  %% Client bibliography here.
  \bibitem{websockets-rfc} IETF, RFC-6455, \emph{The WebSocket Protocol}, \url{http://tools.ietf.org/html/rfc6455} [Data uzyskania dostępu: 28 grudzień 2013]

  \bibitem{caniuse-websockets} \emph{Compatibility table for support of Web Sockets in desktop and mobile browsers}, \url{http://caniuse.com/websockets} [Data uzyskania dostępu: 28 grudzień 2013]

  \bibitem{http-rfc} IETF, RFC-2616, \emph{Hypertext Transfer Protocol - HTTP/1.1} , \url{http://www.ietf.org/rfc/rfc2616.txt} [Data uzyskania dostępu: 28 grudzień 2013]

  \bibitem{xhr-rfc} W3C, Working Draft \emph{XMLHttpRequest} , \url{http://www.w3.org/TR/XMLHttpRequest/} [Data uzyskania dostępu: 28 grudzień 2013]

  \bibitem{touch-events-w3c} W3C Recommendation, {\em Touch Events}, \url{http://www.w3.org/TR/touch-events/} [Data uzyskania dostępu: 22 grudzień 2013]
  
  \bibitem{browser-ios-safari} {\em Safari Web Content Guide}, \url{https://developer.apple.com/library/ios/documentation/AppleApplications/Reference/SafariWebContent/HandlingEvents/HandlingEvents.html} [Data uzyskania dostępu: 22 grudzień 2013]
  
  \bibitem{browser-firefox} Mozilla Developers Network, {\em Web developer guide}, \url{https://developer.mozilla.org/en-US/docs/Web/Guide} [Data uzyskania dostępu: 22 grudzień 2013]
  
  \bibitem{browser-ie} Internet Explorer 10 Guide for Developers, \url{http://msdn.microsoft.com/en-us/library/ie/hh673557(v=vs.85).aspx} [Data uzyskania dostępu: 22 grudzień 2013]
  
  \bibitem{caniuse-touch-events} Compatibility tables for support of HTML5, CSS3, SVG and more in desktop and mobile browsers, {\em Touch Events}, \url{http://caniuse.com/\#feat=touch} [Data uzyskania dostępu: 22 grudzień 2013]
  
  \bibitem{browser-ie} Internet Explorer 10 Guide for Developers, http://msdn.microsoft.com/en-us/library/ie/hh673557(v=vs.85).aspx [Data uzyskania dostępu: 22 grudzień 2013]
  
  \bibitem{caniuse-touch-events} Compatibility tables for support of HTML6, CSS3, SVG and more in desktop and mobile browsers, {\em Touch Events}, http://caniuse.com/\#feat=touch [Data uzyskania dostępu: 22 grudzień 2013]
  
  \bibitem{jquery-mobile} jQuery Mobile 1.4 API Documentation, http://api.jquerymobile.com/category/events/ [Data uzyskania dostępu: 22 grudzień 2013]
  
  \bibitem{jquery-mobile} jQuery Mobile 1.4 API Documentation, \url{http://api.jquerymobile.com/category/events/} [Data uzyskania dostępu: 22 grudzień 2013]
  
  \bibitem{webstorage} W3C Recommendation, {\em Web Storage}, \url{http://www.w3.org/TR/webstorage/}
  
  \bibitem{webstorage-benchmark} Peter McLachlan, \emph{Smartphone Browser localStorage is up to 5x Faster than Native Cache (New Research)}, \url{http://www.mobify.com/blog/smartphone-localstorage-outperforms-browser-cache/} [Data uzyskania dostępu: 23 grudzień 2013]

  \bibitem{http-cache-mobile-benchmark} Tammy Everts, \emph{Mobile browser cache persistence and behaviour}, \url{http://www.webperformancetoday.com/2012/07/12/early-findings-mobile-browser-cache-persistence-and-behaviour/} [Data uzyskania dostępu: 23 grudzień 2013]

  \bibitem{caniuse-webstorage} \emph{Compatibility table for support of Web Storage in desktop and mobile browsers}, \url{http://caniuse.com/namevalue-storage} [Data uzyskania dostępu: 25 grudzień 2013]
  
  \bibitem{rwd} Than Marcotte, \emph{Responsive Web Design}, 25 maja 2010, \url{http://alistapart.com/article/responsive-web-design} [Data uzyskania dostępu: 26 grudzień 2013]

  \bibitem{css21} W3C Recommendation, \emph{Cascading Style Sheets Level 2 Revision 1 (CSS 2.1) Specification}, \url{http://www.w3.org/TR/CSS21/media.html} [Data uzyskania dostępu: 26 grudzień 2013]

  \bibitem{css3} W3C Recommendation, \emph{Cascading Style Sheets (CSS) Snapshot 2010, CSS Level 3}, \url{http://www.w3.org/TR/2011/NOTE-css-2010-20110512/\#css3} [Data uzyskania dostępu: 26 grudzień 2013]

  %% hardware infrastructure bibliography here
  \bibitem{acm}Fitzpatrick, J.: \emph{An interview with Steve Furber: Communications of the ACM}, 2011, t.54, s34-39.
  
  \bibitem{arm-mmu}MMU: \emph{ARM Architecture Documentation}, \url{http://infocenter.arm.com/help/index.jsp?topic=/com.arm.doc.ddi0198e/Babegida.html}

  \bibitem{linux-embedded}Marcin Bis: \emph{Linux w systemach embedded}, Warszawa: btc, 2011

  \bibitem{apple-cocoa} \emph{Quartz Event Service Reference}, \url{https://developer.apple.com/library/mac/documentation/Carbon/Reference/QuartzEventServicesRef/Reference/reference.html}
  \bibitem{reactive-programming} \emph{The Reactive Manifesto}, \url{http://www.reactivemanifesto.org/}
  \bibitem{learning-play}Andy Petrella: \emph{Learning Play! Framework 2}, Birmingham: Packt, 2013 
\end{thebibliography}
\listoftables

\listoffigures












\begin{appendices}

\section{Zawartość CD}

The contents...

\end{appendices}


\end{document}
