\section{Serwer}

% o komunikacji ogólnie w projekcie

\subsection{Komunikacja między klientem webowym, a serwerem}

% o Comlet

\subsubsection{Komunikacja HTTP persistent connection}

% http://en.wikipedia.org/wiki/HTTP_persistent_connection

\subsubsection{Pushlet, JSONP Pooling oraz Script Tag long pooling}

% http://en.wikipedia.org/wiki/Push_technology#Pushlet
% http://en.wikipedia.org/wiki/Comet_(programming)#Hidden_iframe
% http://en.wikipedia.org/wiki/JSONP

\subsubsection{Bidirectional-streams Over Synchronous HTTP (BOSH)}

% http://en.wikipedia.org/wiki/BOSH

\subsubsection{XHR long pooling}

% http://en.wikipedia.org/wiki/Comet_(programming)#XMLHttpRequest_long_polling

\subsubsection{Metody oparte o wykorzystanie Flash i Java Applets}

% 

\subsubsection{Web Sockets}

% 

\subsection{Wykorzystana technologia}

\subsubsection{Synchroniczny i imperatywny model programowania}

\subsubsection{Asynchroniczny i funkcyjny model programowania}

\subsubsection{Zasada działania Node.js}

\subsubsection{Socket.io}

\subsubsection{Skalowanie}

% http://book.mixu.net/node/ch13.html