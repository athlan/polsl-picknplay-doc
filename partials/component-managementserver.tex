\subsection{Zarządzania infrastrukturą za pomocą aplikacji internetowej}

Zgodnie z diagramem znajdującym się w załączniku ~\ref{app:network_diagram_app}, serwer www zarządzający infrastrukturą niekoniecznie musi być umieszczony na tym samym urządzeniu co serwer wyświetlania, z racji tego serwer www musi sterować tym serwerem przez protokół ssh\footnote{ang. \emph{Secure Shell} - to standard protokołów komunikacyjnych używanych w sieciach komputerowych TCP/IP, w architekturze klient-serwer.\cite{ssh-wiki}}.
Do logowania do zdalnej stacji roboczej na serwerze www użyto biblioteki \lstinline{scala-ssh}. Poniższy fragment kodu źródłowego przedstawia metode obsługującą logowanie do aplikacji webowej.

\begin{lstlisting}
def signin() = Action.async(parse.json) {
    implicit request =>
    {
      var login : String = ""
      var f = Future[Boolean] {
        var sshIPAddress = (request.body \ "ip_address").asOpt[String]
        var someLogin = (request.body \ "login").asOpt[String]
        var password = (request.body \ "password").asOpt[String]
        if (sshIPAddress.isEmpty || someLogin.isEmpty || password.isEmpty) {
          false
        }
        else {
          login = someLogin.get
          SSHUser.connect(sshIPAddress.get, someLogin.get, password.get)
        }
      }
      f.map(fut => {
        var res = Ok(toJson(Map("success" -> fut)))
        if (fut) {
          res.withSession(Security.username -> login)
        }
        else
          res
      })
    }
  }
\end{lstlisting}

Metoda odpowiada w formacie JSON\footnote{ang. \emph{JavaScxript Object Notation} - lekki format wymiany danych.}, gdyż żądanie, które wysyła przeglądarka internetowa, jest wysłane w technice AJAX \footnote{ang. \emph{Asynchronous JavaScript and XML.}}. Dane, które zostają wysłane do serwera, również są w formacie JSON, po wyłuskaniu danych do odpowiednich zmiennych w linii 14 wykonywana jest próba logowania do zdalnej stacji roboczej. Cała metoda \lstinline{signin} jest asynchroniczna, przez co wykonanie metody \lstinline{connect} nie blokuje wątku. Dzięki takiemu podejściu wątek, który obecnie czeka na odpowiedź z serwera, może obsłużyć kolejne żądanie.

\par

Po pomyślnym zalogowaniu użytkownik może dodać nowych klientów oraz uruchomić na nich aplikacje wybierając je z listy.

\begin{lstlisting}
  def runXApp(app: PNPProcessApp) {
    currentProcesses.foreach(p => p.kill)
    if (app.name != PNPProcessApp.none.name) {
      SSHUser.sshClient.exec(app.toString + " -display " + ip_screen_address)
    }
  }
\end{lstlisting}

Powyższa metoda jest wywołana za każdym razem gdy użytkownik przełączy aplikację. W linii drugiej 'zabijana' jest dotychczasowo uruchomiona aplikacja, a następnie przy pomocy biblioteki \lstinline{scala-ssh} wywoływana jest aplikacja wybrana z listy, z parametrem display określającym miejsce wyświetlenie aplikacji.



