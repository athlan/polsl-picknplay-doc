\newpage
\section{Założenia wstępne, określenie wymagań}
\label{sec:requirement}

\subsection{Wymagania funkcjonalne}

Przed przystąpieniem do prac sformułowane następujące możliwości systemu:

\begin{itemize}
	\item użytkownik powinien mieć możliwość sterowania myszką z poziomu przeglądarki internetowej swojego telefonu komórkowego po podłączeniu się do sieci,
	\item administrator systemu powinien mieć możliwość kontroli uruchomionych aplikacji na ekranach z poziomu przeglądarki internetowej,
	\item użytkownik powinien mieć możliwość wzięcia udziału w grze z innymi użytkownikami podłączonymi do systemu.
\end{itemize}


\subsection{Wymagania niefunkcjonalne}

Określone również zostały wymagania jakościowe:

\begin{itemize}
	\item system dystrybucji obrazu nie powinien być powiązany z żadnym konkretnym producentem ekranów, jedynym wymaganiem co do wyświetlacza jest posiadanie złącza HDMI,
	\item dodanie nowego ekranu do infrastruktury powinno być łatwe i mało kosztowne,
	\item system powinien uwzględniać duże rozproszenie infrastruktury,
	\item system powinien mieć możliwość wyświetlenia na ekranie zarówno przeglądarki internetowej z obsługą technologii Flash, jak i aplikacji natywnej dla systemu operacyjnego.
\end{itemize}
