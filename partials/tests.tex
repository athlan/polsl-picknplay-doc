\newpage
\section{Testowanie i uruchamianie}

Testowanie i uruchamianie to konieczna część przebiegu prac. Z racji tego, że stworzony projekt nie ma specyfiki monolitycznego programu, a wręcz rozproszenie kodu pomiędzy systemami jest znaczne, w głównej mierze dział skoncentrowany jest na  zstępujących testach integracyjnych w punktach zestyku komponentów.

Wykonane testy zakładały zarówno poprawną, jak i niepoprawną konfigurację systemu.

Istotnym dla pilotów sterujących były testy funkcjonalne na urządzeniach mobilnych o ekranach różnej rozdzielczości oraz różnych systemach operacyjnych w starszych i nowszych wersjach. Urządzenia mobilne, na których zostały przeprowadzone testy, to:
\begin{itemize}
	\item iPhone 5, system operacyjny: iOS 7.0.3
	\item iPhone 4, system operacyjny: iOS 4.1.0 (brak obsługi Web Sockets)
	\item iPad, system operacyjny: iOS 3.1.2 (brak obsługi Web Sockets)
	\item Samsung Galaxy S III, system operacyjny: Android 4.1
	\item HTC Wildfire S, system operacyjny: Android 2.3.5 (brak obsługi Web Sockets)
\end{itemize}

\subsection{Pilot do sterowania zdalnym monitorem}

Przetestowane zostały:

\begin{itemize}
	\item poprawność wyświetlania się elementów strony internetowej pilota na urządzeniach mobilnych o różnych rozdzielczościach ekranu,
	\item poprawność przechwytywania i interpretacji punktów styku użytkownika z ekranem urządzenia mobilnego,
	\item poprawność działania komunikacji klienta pilota z serwerem i vice versa, sposób reagowania na otrzymywane komunikaty wg specyfikacji interfejsów,
	\item poprawność łączenia się serwera do programu generującego zdarzenia urządzeń wejścia oraz poprawność przesyłanych danych.
\end{itemize}

Wszystkie testy zostały pomyślnie zakończone.

\subsection{Gra PONG}

Przetestowane zostały:

\begin{itemize}
	\item poprawność wyświetlania się elementów strony internetowej pilota na urządzeniach mobilnych o różnych rozdzielczościach ekranu,
	\item poprawność przechwytywania i interpretacji punktów styku użytkownika z ekranem urządzenia mobilnego,
	\item poprawność działania komunikacji klienta z serwerem i vice versa, sposób reagowania na otrzymywane komunikaty wg specyfikacji interfejsów,
	\item poprawność łączenia się serwera do klienta wyświetlającego grę,
	\item poprawność działania klienta wyświetlającego grę, jego wyświetlania i logiki.
\end{itemize}

Wszystkie testy zostały pomyślnie zakończone, aczkolwiek sporo czasu zajęło dostosowanie wyglądu pilota pod wszystkie urządzenia mobilne. Urządzenia mobilne bez wsparcia protokołu Web Sockets generowały opóźnienia w komunikacji, stosując alternatywne metody komunikacji HTTP Comet.

\subsection{Program generujący zdarzenia urządzeń wejścia}
Przetestowane zostały:
\begin{itemize}
	\item poprawność odbierania komunikatów, oraz przetwarzania ich na systemach operacyjnych Linux, oraz Mac OS X,
	\item poprawność określania rozdzielczości ekranu dla systemów Linux i Mac OS X.
\end{itemize}

\subsection{Serwer zarządzania infrastrukturą}

Przetestowane zostały:
\begin{itemize}
	\item poprawność logowania, oraz wylogowywania na zdalny host przez klienta ssh,
	\item poprawność przełączania aplikacji,
	\item poprawność wyłączania uruchomionej aplikacji.
\end{itemize}

