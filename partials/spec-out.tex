%\section{Specyfikacja zewnętrzna}

%\section{Przypadki użycia}

\section{Uruchamianie systemu}

\subsection{Uruchomienie serwera node.js}

Do uruchomienia serwera niezbędna jest instalacja platformy node.js w odpowiedniej wersji. W tym celu należy ściągnąć kod źródłowy z oficjalnego repozytorium (\url{https://github.com/joyent/node} oznaczony tagiem 0.10.12. Po ściągnięciu i rozpakowaniu paczki należy skompilować i zainstalować platformę poleceniami:

\begin{lstlisting}
./configure
make
make install
\end{lstlisting}

Po pomyślnej instalacji należy ściągnąć repozytorium projektu dostępne po adresem - \url{https://github.com/athlan/remotely}. By rozwiązać wszystkie zależności projektu należy w katalogu z projektem wykonać polecenie \lstinline{npm install}.
\par
Do uruchomienia aplikacji pozostaje jeszcze uruchomić serwer sieciowy. W załączniku ~\ref{app:apache} omówiona została instalacja, oraz konfiguracja serwera Apache.
\par
Przed ostatnim krokiem w konfiguracji serwera jest wyedytowanie pliku index.html w głównym katalogu projektu remotely. Atrybuty przy znaczniku body to:
\begin{description}
	\item[data-server] \hfill \\
		adres komputera wraz z numerem portu, na którym jest uruchomiony serwer sieciowy
	\item[data-server-remote-host] \hfill \\
		adres klienta na który wysyłane będą zdarzenia urządzeń wejścia.
	\item[data-server-remote-port] \hfill \\
		numer portu klienta
\end{description}
\par
Po skonfigurowaniu serwera pozostaje go tylko uruchomić poleceniem \lstinline{node server.js}.


\subsection{Uruchomienie klienta Linux}

Aby uruchomić oprogramowanie klienta Linux należy pobrać repozytorium dostępne pod adresem: \url{https://github.com/athlan/remotely}, a następnie w folderze client wybrać dostępny system (macosx lub linux), i w odpowiednim podfolderze skompilować oprogramowanie poleceniem \lstinline{make}.
Po pomyślnej kompilacji należy uruchomić serwer poleceniem \lstinline{./bin/server}.


\subsection{Uruchomienie serwera www zarządzającego infrastrukturą}

W celu zainstalowania serwera www zarządzającego infrastrukturą należy zainstalować framework play, a następnie pobrać z repozytorium kod źródłowy aplikacji zarządzającej systemem dostępy pod adresem - \url{https://github.com/badeleux/remotely-web}. Uruchomienie serwera za pomocą polecenia \lstinline|play ~run|, rozwiąże wszystkie zależności, oraz skompiluje kod. 


