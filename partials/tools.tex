\section{Narzędzia}
\label{cha:Narzędzia}

\subsection{GIT}
\label{sub:GIT}
Rozproszony system kontroli wersji. System ten pozwolił nam na bezproblemową synchronizacje naszej pracy między stacjami roboczymi. Dzięki temu, że jest on rozproszony nie jest konieczne wyznaczanie głównego serwera przez, który synchronizujemy naszą pracę, ale każdy z klientów może pełnić rolę serwera do którego wysyłamy zmiany plików objętych systemem kontroli wersji.


\subsection{Vim}
\label{sub:Vim}
Prawdopodobnie najpopularniejszy konsolowy edytor tekstu. Vim jest klonem edytora vi, z wieloma nowymi funkcjami. Przede wszystkim jest to modalny edytor tesktu tzn. że posiada więcej niż jeden tryb pracy (INPUT, COMMAND, VISUAL). Dzięki swojej prostocie, szybkości, oraz mnogości wtyczek ten konsolowy edytor jest wciąż bardzo popularny i używany w zastępstwie wielu zintegrowanych środowisk programistycznych jak np. Eclipse czy VisualStudio.  


