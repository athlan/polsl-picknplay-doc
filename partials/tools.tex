\section{Narzędzia}
\label{cha:Narzędzia}

\subsection{Narzędzia wspomagające zarządzanie projektem}

\subsubsection{GIT}
\label{sub:GIT}
Rozproszony system kontroli wersji. System ten pozwola na bezproblemową synchronizacje pracy między stacjami roboczymi. Dzięki temu, że jest on rozproszony nie jest konieczne wyznaczanie głównego serwera, przez który synchronizowany jest kod źródłowy, a każdy z klientów może pełnić rolę serwera do którego wysyłane są zmiany plików objętych systemem kontroli wersji.

Przy realizacji pracy dyplomowej GIT pomógł rozwiązywać konflikty w treści kodu źródłowego, ciągłe nadpisywanie treści plików powstałe w wyniku pracy kilku osób. Do przechowania kodu źródłowego na serwerze zdalnym wykorzystana została darmowa powierzchnia serwisów GitHub oraz Bitbucket. Na Githubie umieszczone zostały aplikacje stworzone w ramach pracy dyplomowej oraz książka napisana w LaTex.


\subsubsection{Asana}
\label{sub:Asana}

Z racji na sposobność pracy w zespole nad pracą dyplomową niezbędne było narzędzie wspomagające komunikację oraz kontrolę wykonywania zadań. Asana to system zarządzania projektem umożliwiający rozdzielanie zadań pomiędzy członków zespołu przypisanych do projektu, śledzenie postępów prac, ustalania przypomnień oraz terminów. System wspomaga komunikację, dzieli obszary zainteresowań na podprojekty, umożliwia oznaczenie zagadnień słowami kluczowymi oraz udostępnia wyszukiwarkę. Narzędzie dostępne w modelu SaaS (ang. Software as a Service) dostępne przez przeglądarkę internetową bez konieczności instalacji oprogramowania.

W projekcie narzędzie wykorzystane zostało do monitorowania postępów prac oraz tworzenie bazy wiedzy.

\subsection{Narzędzia programistyczne}

\subsubsection{Vim}
\label{sub:Vim}
Prawdopodobnie najpopularniejszy konsolowy edytor tekstu. Vim jest klonem edytora vi, z wieloma nowymi funkcjami. Przede wszystkim jest to modalny edytor tesktu tzn. że posiada więcej niż jeden tryb pracy (INPUT, COMMAND, VISUAL). Dzięki swojej prostocie, szybkości, oraz mnogości wtyczek ten konsolowy edytor jest wciąż bardzo popularny i używany w zastępstwie wielu zintegrowanych środowisk programistycznych jak np. Eclipse czy VisualStudio.  
