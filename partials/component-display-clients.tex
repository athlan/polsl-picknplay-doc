\subsection{Program generujący zdarzenia urządzeń wejścia}
\label{sub:impl-displayclient-events-dispatcher}

\subsubsection{Linux}
Aplikacja dla systemu Linux składa się z następujących komponentów: 
\begin{description}
	\item[serwer] \hfill \\
Kod źródłowy serwera znajduje się w załączniku ~\ref{app:server_c}, jest on wspólny dla wszystkich platform unixowych.		
	\item[generator zdarzeń wejścia] \hfill \\ 
		Program generujący zdarzenia urządzeń wejścia na podstawie danych otrzymanych z serwera. W platformie Linux wykorzystano moduł jądra \lstinline{uinput}. Uinput to moduł pozwalający na generowanie zdarzeń symulujących urządzenia wyjścia. Moduł ten tworzy specjalny plik urządzenia w katalogu \lstinline{/dev}. Plik ten jest wirtualnym interfejsem, co znaczy, że nie jest połączony z żadnym fizycznym urządzeniem.

		Jednym z zadań generatora jest ustalenie rozdzielczości ekranu, na którym wyświetlany jest obraz. Informacja ta jest niezbędna, gdyż koordynaty wysyłane przez serwer są podawane absolutnie, oraz z przedziału od 0 do 1 (ze względu na to, że informacja ta nie może być uzależniona od rozdzielczości monitora). Na platformie Linux rozdzielczość uzyskiwana jest za pomocą następującego polecenia:
\begin{lstlisting}[language=bash]
"xdpyinfo -display :0 | sed 's/^ *dimensions: *\\([0-9x]*\\).*/\\1/;t;d'";
\end{lstlisting}
		Program xdpyinfo jest to aplikacja służąca do wyświetlania informacji o serwerze X, z kolei za pomocą wyrażenia regularnego i programu sed wyłuskano tylko niezbędne informacje dotyczące rozdzielczości ekranu.

		Istotnym elementem w pisaniu programu generującego zdarzenia urządzeń wejścia jest odbiór danych z serwera, opisanego w załączniku ~\ref{app:server_c}. Program ten wysyła dane do deskryptora pliku o nazwie \lstinline{STDIN_FILENO}. Nie jest to bynajmniej to samo, co wskaźnik do pliku o nazwie \lstinline{stdin}, przez co nie jest możliwe użycie funkcji wysokiego poziomu, takiej jak na przykład: \lstinline{scanf}, do wczytania danych. Niezbędne jest użycie funkcji \lstinline{read}, a następnie użycie funkcji \lstinline{sscanf} w celu sformatowaniu danych, co pokazuje poniższy fragment kodu:

\begin{lstlisting}[language=c]
read(STDIN_FILENO, buffer, sizeof(buffer));
int result = sscanf(buffer, "x:%lf:y:%lf;", &xTmp, &yTmp);
int x = xTmp * dim[0];
int y = yTmp * dim[1];
\end{lstlisting}

Wczytane koordynaty należy wysłać jako parametry zdarzeń do modułu \lstinline{uinput}. 

\begin{lstlisting}[language=c]
#include <linux/input.h>
#include <linux/uinput.h>
int fd = open("/dev/input/uinput", O_WRONLY | O_NONBLOCK);
ret = ioctl(fd, UI_SET_EVBIT, EV_ABS);

...

struct input_event ev, synEv;


...
memset(&ev, 0, sizeof(ev));
ev.type = EV_ABS;
ev.code = ABS_X;
ev.value = x;
if(write(fd, &ev, sizeof(struct input_event)) < 0)
die("error: write");  

memset(&synEv, 0, sizeof(struct input_event));
synEv.type = EV_SYN;
synEv.code = 0;
synEv.value = 0;
if(write(fd, &synEv, sizeof(struct input_event)) < 0)
die("error: write");

...
//sending y coordinate


\end{lstlisting}
		Na początku należy włączyć niezbędne nagłówki (linie 1-2), następnie otworzyć plik urządzenia \lstinline{uinput}, do którego będą wysyłane zdarzenia urządzeń wejścia. W linii 4 powyższego listingu włączono możliwość wysyłania zdarzeń ruchów myszką, podając koordynaty absolutne. Struktury zadeklarowane w linii 8 zawierają 3 pola: 
\begin{itemize}
\item \lstinline{type} - typ zdarzenia (\lstinline{EV_KEY}, \lstinline{EV_ABS}, \lstinline{EV_REL})		
\item \lstinline{code} - dla zdarzeń klawiatury jest to kod klawisza, natomiast dla zdarzeń myszy jest to oś.
\item \lstinline{value} - dla zdarzeń klawiatury 1 oznacza przyciśnięcie klawisza, a 0 puszczenie, dla zdarzeń myszy jest to z kolei wartość reprezentująca koordynatę.
\end{itemize}
		Po wypełnieniu struktury odpowiednimi wartościami należy wysłać je do wcześniej otwartego pliku (\lstinline{fd}), a następnie wypełnić i wysłać strukturę synchronizującą (\lstinline{evSyn}), której zadaniem jest zatwierdzenie, oraz realizacja wszystkich dotychczasowo wysłanych zdarzeń. 
	\item[serwer X] \hfill \\
	Ta część systemu jest odpowiedzialna za wyświetlanie klientów X.
		Moduł ten opisany jest bardziej szczegółowo w załączniku ~\ref{app:X Window System}.


\end{description}

\subsubsection{Mac OS X}

W niniejszym podrozdziale nie zostanie omówiony proces instalacji systemu Mac OS X na żadnym z jednoukładowych komputerów, gdyż jest to niemożliwe ze względu na to, że system Mac OS X jest dedykowanym oprogramowaniem dla komputerów firmy Apple. 

Serwer odbierający zdarzenia jest wspólny dla wszystkich systemów unixowych i został opisany w załączniku ~\ref{app:server_c}.
\\
Kod generatora zdarzeń dla systemu Mac OS X jest nieco prostszy aniżeli dla systemu Linux. Bazuje on na zdarzeniach z biblioteki CoreGraphics. Tak jak i dla systemu Linux, istotnym elementem jest odczytanie rozdzielczości ekranu. Jest to wykonane za pomocą polecenia:
\begin{lstlisting}
system_profiler SPDisplaysDataType | grep Resolution | sed 's/[^0-9 ]*//g' | sed -e 's/^[ \t]*//' | sed 's/  */x/g'
\end{lstlisting}
Format wyniku powyższego polecenia jest dokładnie taki sam, jak dla instrukcji analogicznej w systemie Linux.
\\
Kod programu odpowiadający za wygenerowanie zdarzenia na podstawie koordynat znajduje się na poniższym listingu:
\begin{lstlisting}
	CGEventRef move1 = CGEventCreateMouseEvent(
	NULL, kCGEventMouseMoved,
	CGPointMake(x, y),
	kCGMouseButtonLeft
);

CGEventPost(kCGHIDEventTap, move1);

CFRelease(move1);
\end{lstlisting}

Funkcja \lstinline{CGEventCreateMouseEvent} tworzy nowe zdarzenie, polegające na przesunięciu kursora myszy do zadanej lokacji.

\subsection{Klient wyświetlający grę PONG na zdalnym monitorze}
\label{sub:impl-displayclient-game}

Klient gry został napisany w JavaScript oraz CSS 3 wykorzystując jego mechanizmy. Zawiera w sobie logikę gry i wysyła komunikaty do serwera sterującego. Wyświetla napis z zaproszeniem do dołączenia do gry bądź (w trakcie jej trwania) paletki wraz z piłką. Klient jest uruchamiany w przeglądarce internetowej wyświetlanej na zdalnym monitorze.

Interfejs wyjściowy dla gry: 

\begin{description}
	\item[signalGameStop] \hfill \\
	Komunikat wysyłany do serwera gry, informujący o jej zakończeniu wraz z danymi pozwalającymi ustalić wynik.
	\begin{enumerate}
		\item scoreL (int) wynik gracza po lewej stornie
		\item scoreR (int) wynik gracza po prawej stronie
	\end{enumerate}
\end{description}

Interfejs wejściowy dla gry:

\begin{description}
	\item[paddleMove] \hfill \\
	Komunikat wysyłany przez serwer gry, informujący o zmianie położenia paletki gracza.
	\begin{enumerate}
		\item positionX (float)
		\item positionY (float)
		\item paddleId (string) identyfikator paletki (\emph{l} - lewa, \emph{r} - prawa)
	\end{enumerate}
\end{description}

\begin{description}
	\item[signalGameSart] \hfill \\
	Komunikat wysyłany przez serwer gry, informujący grę o konieczności jej zakończenia (z przyczyny np. ucieczki jednego z gracza tj. rozłączenia się).
\end{description}

\begin{description}
	\item[signalGameStop] \hfill \\
	Komunikat wysyłany przez serwer gry, informujący grę o konieczności jej inicjalizacji oraz rozpoczęcia.
\end{description}
