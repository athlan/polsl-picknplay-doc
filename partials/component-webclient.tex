\section{Klient webowy - pilot uruchomiony na urządzeniu mobilnym}

Kluczowym z punktu widzenia systemu jest poprawnie działający, uniwersalny klient webowy w postaci strony internetowej wyświetlanej w przeglądarce na urządzeniu mobilnym klienta końcowego. Biorąc pod uwagę różnorodność rozdzielczości ekranów, gęstości pikseli (ang. \emph{pixel density}), szybkości wykorzystywanych łącz, wsparcia przeglądarek internetowych na urządzeniach mobilnych dla dyrektyw CSS 3\footnote{CSS (ang. \emph{Cascading Style Sheets}) kaskadowe arkusze stylów, opisujące sposób wyświetlania elementów strony internetowej)}, SVG\footnote{SVG (ang. \emph{Scalable Vector Graphics}) format grafiki wektorowej} rozpoznano szereg problemów oraz zaproponowano rozwiązania.

\subsection{Projektowanie stron internetowych zgodnie z Responsive Design}



\subsection{Wykrywanie gestów telefonu komórkowego}

\lstset{language=JavaScript}
\begin{lstlisting}[label=some-code,caption=Some Code]
var test = function(param) {
	// obsluga gestow
}
\end{lstlisting}

\subsubsection{Reprezentacja kursora na płaszczyźnie}

Pozycja kursora reprezentowana jest przez punkt na dwuwymiarowej płaszczyźnie w zakresie \( x\in \langle0, 1\rangle \).

\subsection{Pamięć podręczna Web Storage}
