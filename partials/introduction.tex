\newpage
\section{Wprowadzenie}

Nowe technologie oraz wzrost popularności urządzeń mobilnych (ang. \emph{smartphone}) otwierają nowy obszar działań w dziedzinie marketingu, a w szczególności kreowania i wzmacniania marki w świadomości konsumentów. Odpowiednio przygotowany marketing wirusowy (ang. 
\emph{viral marketing}) poprzez rozpowszechnianie informacji przez samych potencjalnych klientów, może przyczynić się do dużego zasięgu informacji na temat marki (na przykład w mediach społecznościowych) zerowym kosztem dalszej propagacji informacji. Działania te mogą bezpośrednio wpłynąć na wysoką konwersję sprzedaży w kontekście dostarczenia informacji konsumentowi końcowemu od znajomego, z polecenia, co buduje większe zaufanie. W połączeniu z coraz bardziej popularną techniką angażowania konsumenta - grywalizacją (inaczej grywalizacją, ang. \emph{gamification}) w sposób dla niego przyjemny, wspomaga budowanie wizerunku marki.

Ostatecznym efektem działań w ramach pracy dyplomowej jest stworzenie prototypu technologii, która może być wykorzystana jako gra marketingowa wyświetlana na ekranie LED umieszczonym w centrum miasta lub w klubie, sterowana bezpośrednio ze smartfona bez konieczności instalacji jakiejkolwiek aplikacji dedykowanej na dany system operacyjny, aby zminimalizować próg wejścia do gry do niezbędnego minimum (np. do wpisania adresu w przeglądarkę internetową lub zeskanowania QR Code [ang. \emph{Quick Response Code}]).

Możemy wyobrazić sobie sytuację, że na mieście pojawia się ekran LED, na którym przechodnie mogą wysterować grą miejską angażującą klientów danej marki w postaci zdobycia punktów z możliwością ich wymiany lub osiągnięciem statusu ambasadora marki. Informacja rozchodzi się po sieci w mediach społecznościowych w postaci nagrań filmów, natomiast firma inicjująca formę grywalizacji bezkosztowo propaguje informacje utwierdzające jej istnienie wśród konsumentów.

Korzystając z faktu powstania technologii WebSocket protocol (ustandaryzowanej przez IETF\footnote{Internet Engineering Task Force} w 2011 roku w dokumencie RFC\footnote{Request for Comments} 6455\cite{websockets-rfc}) umożliwiającą dwukierunkową transmisję danych pomiędzy przeglądarką internetową oraz serwerem poprzez gniazdo (ang. \emph{socket})  wspieraną przez większość przeglądarek zainstalowanych na urządzeniach mobilnych. Intencją pracy dyplomowej jest stworzenie mechanizmu sterowania komputerem z poziomu przeglądarki internetowej uruchomionej na urządzeniu mobilnym bez konieczności instalacji zewnętrznego oprogramowania w postaci aplikacji dedykowanej dla danego systemu operacyjnego. Korzystając również z faktu, iż aktywni użytkownicy urządzeń mobilnych chętnie aktualizują oprogramowanie, wprowadzenie tej technologii zwiększa zasięg użytkowników, którzy mogą skorzystać z technologii z każdą aktualizacją (200 milionów uaktualnionych urządzeń firmy Apple\footnote{Statystyki z oficjalnych press release firmy Apple, uaktualnienie systemu operacyjnego do iOS7 http://www.apple.com/pr/library/2013/09/23First-Weekend-iPhone-Sales-Top-Nine-Million-Sets-New-Record.html} oraz pokrycia ponad 55,6 proc. Androida w wersji 4.1 i powyżej \footnote{Statystyki z dnia 02.12.2013 r. http://developer.android.com/about/dashboards/index.html}).

Realizacja pracy dyplomowej uwzględnia również optymalizację rozmiaru oraz sposobu ładowania kodu do przeglądarki internetowej na urządzeniu mobilnym mając na uwadze dostęp do wolniejszej sieci 3G (niż w przypadku dostępu do szerokopasmowej sieci Internet lub LTE [ang. \emph{Long Term Evolution}, nazywany inaczej 4G LTE]). W celu szybszego ponownego (przy kolejnych odwiedzinach) ładowania statycznej treści wykorzystany został \emph{Local Sotorage}\cite{webstorage} wprowadzony wraz z HTML5, ustandaryzowany przez W3C\footnote{World Wide Web Consortium} w dokumencie W3C Recommendation Web Storage, a do serwowania statycznej treści po raz pierwszy został zbudowany lokalny CDN (ang. \emph{Content Delivery Network}) złożony z Varnish Cache.

\subsection{Przewodnik po działach pracy}

Praca została podzielona na 13 rozdziałów, według ogólnie przyjętego wzorca. 

\par

W rozdziałach ~\ref{sec:goal} i ~\ref{sec:requirement} został omówiony cel, oraz wymagania jakie zostały postawione zanim przystąpiono do prac budowania systemu. Jest to istotny element w budowie każdego systemu - ściśle postawione wymagania są fundamentem analizy każdego systemu informatycznego.

\par
W rozdziale ~\ref{sec:analyse} omówiono zostały problemy z jakimi się spotkano pisząc pracę, uzasadnienie wybranych technologii, oraz rozwiązań w każdym z poszczególnych komponentów.

\par

W części numer ~\ref{sec:tools} zaprezentowano narzędzia jakimi się posługiwano tworząc system, w początkowych podrozdziałach opisano ogólne instrumenty, które używano we wszystkich komponentach aplikacji, natomiast kolejne podrozdziały mówią o poszczególnych częściach i istotnych środkach użytych w celu uzyskania oczekiwanego rezultatu.

\par

W specyfikacji zewnętrznej (rozdział numer  ~\ref{sec:spec-out}) omówiono system ze strony zwykłego użytkownika na przykładzie gry Pong, jak i część systemu od strony administratora zarządzającego infrastrukturą monitorów. Zamieszczone zostały zrzuty ekranu, oraz krótki ich opis.

\par

Rozdział numer ~\ref{sec:implementation} opisuje wybrane problemy i fragmenty kodu istotne dla tej pracy wraz z opisem używanych funkcji i mechanizmów.

\par

Kolejne podrozdziały omawiają uruchomienie, oraz instalacja systemu niezbędne do poprawnego działania systemu. 

\par

Dodatek ~\ref{app:network_diagram_app} jasno pokazuje fizyczną architekturę aplikacji.
Dodatki ~\ref{app:pong_comp_diagram}, oraz ~\ref{app:control_comp_diagram} prezentują komponenty oprogramowania wykorzystanego w systemie.
Dodatek ~\ref{app:server_c} przedstawia kod serwera użytego na wszystkich platformach UNIX-owych właśnie z tego powodu jego treść, oraz omówienie zostało umieszczone w dodatku, gdyż odnosi się do kilku rozdziałów.
Dodatek ~\ref{app:X Window System} jest krótkich przedstawianiem działania systemu X'ów, nie jest to kompletny przewodnik, ale raczej krótkie uzmysłowienie działania tego oprogramowania.
Dodatek ~\ref{app:apache} przedstawia instalacje serwera internetowego dla systemu Linux Debian. 

\subsection{Podział prac}

Pan Kamil Badyla odpowiedzialny był za pracę nad oprogramowaniem, konfiguracją, oraz analizą części systemu odpowiedzialnej za wyświetlanie obrazu na rozproszonych monitorach, oraz aplikacji internetowej zarządzającej systemem.

Pan Piotr Pelczar odpowiedzialny był za pracę nad zaprojektowaniem ogólnej architektury systemu, napisaniem pilotów sterujących oraz serwerów, do których łączą się piloty. Napisana została również przykładowa gra PONG oraz przeprowadzona została również dywagacja na temat możliwości rozproszenia systemu.

Wspólna praca to integracja systemów, dokumentacja, testowanie i uruchamianie oraz wyciąganie wniosków.
