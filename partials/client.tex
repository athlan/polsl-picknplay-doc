\section{Klient}
\label{sec:Klient}

Klient to komputer podłączony bezpośrednio do ekranu wyświetlającego obraz. Istotnymi cechami takiego urządzenia są przede wszystkim:
\begin{itemize}
	\item niska cena
	\item wystarczająca moc obliczeniowa
	\item dostępność
	\item mały rozmiar
\end{itemize}

Taka lista wymagań zawęża mocno zbiór urządzeń dostępnych na rynku. Komputerem spełniającym powyższe kryteria, oraz cieszącym się sporą popularnością, dzięki czemu też nie małym wsparciem technicznym jest platforma BeagleBone Black. Komputer ten wyposażony jest w procesor ARM - Cortex A8 taktowany zegarem 1GHz, i pamięcią operacyjną 512MB DDR3, co pozwala swobodne zainstalowanie Linuksa. 

\subsection{System operacyjny}
\label{sub:System operacyjny}
Nie dysponując zbyt dużą mocą obliczeniową, małą pamięcią operacyjną, oraz niewielką przestrzenią dyskową należy dobrze dobrać, oraz skonfigurować system operacyjny. Do tego celu wybrano dystrybucje Linuksa - Debian. Cechuje się ona przede wszystim wysoką stabilnością (bardzo często wybierana jako system serwerowy), sporą społecznością użytkowników co przekłada się na mnogość dostępnych pakietów przygotowanych specjalnie dla tej wersji systemu Linux. Wreszcie system Debian jest wysoce konfigurowalny - oparty o jądro Linux, daje możliwość zbudowania systemu od podstaw.


