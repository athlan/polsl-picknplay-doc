\newpage
\section{Instalacja i uruchamianie systemu}

\subsection{Uruchomienie serwera node.js}
\label{subsub:setup-server-nodejs}

Do uruchomienia serwera niezbędna jest instalacja platformy node.js w odpowiedniej wersji. W tym celu należy ściągnąć kod źródłowy z oficjalnego repozytorium (\url{https://github.com/joyent/node} oznaczony tagiem 0.10.12. Po ściągnięciu i rozpakowaniu paczki należy skompilować i zainstalować platformę poleceniami:

\begin{lstlisting}
./configure
make
make install
\end{lstlisting}

Po pomyślnej instalacji należy ściągnąć repozytorium projektu dostępne po adresem - \url{https://github.com/athlan/remotely}. By rozwiązać wszystkie zależności projektu należy w katalogu z projektem wykonać polecenie \lstinline{npm install}.
\par
Do uruchomienia aplikacji pozostaje jeszcze uruchomić serwer sieciowy. W załączniku ~\ref{app:apache} omówiona została instalacja, oraz konfiguracja serwera Apache.
\par
Przed ostatnim krokiem w konfiguracji serwera jest wyedytowanie pliku index.html w głównym katalogu projektu remotely. Atrybuty przy znaczniku body to:
\begin{description}
	\item[data-server] \hfill \\
		adres komputera wraz z numerem portu, na którym jest uruchomiony serwer sieciowy
	\item[data-server-remote-host] \hfill \\
		adres klienta na który wysyłane będą zdarzenia urządzeń wejścia.
	\item[data-server-remote-port] \hfill \\
		numer portu klienta
\end{description}
\par
Po skonfigurowaniu serwera pozostaje go tylko uruchomić poleceniem \lstinline{node server.js}.


\subsection{Przygotowanie bazowej wersji systemu Linux wyświetlającego obraz na ekranie}


Instalując dystrybucję linuksa na komputerze PC zwykle nie ma większego sensu by kompilować jądro. Większość dystrybucji udostępnia gotowy obraz, który zawiera podstawowe programy i usługi niezbędne do instalacji systemu operacyjnego z środowiskiem graficznym. W przypadku systemu wbudowanego wygląda to nieco inaczej. Nie ma jednego gotowego obrazu pasującego do wszystkich jednoukładowych komputerów. Dlatego też w niniejszym rozdziale opisana zostanie instalacja systemu Linux od podstaw \footnote{ang. \emph{Linux From Scratch}}.

\par


Na początku należy stworzyć sobie miejsce na dysku, w którym system będzie budowany - \lstinline{mkdir ~/ARM_DEBIAN && alias arm='cd ~/arm' && arm}.

\subsubsection{Przygotowanie karty SDHC}


System debian zostanie zainstalowany na karcie SDHC \footnote{ang. \em{Secure Digital High Capacity}}, po czym zostanie uruchomiony na urządzeniu Rickomagic MK802 II. Komputer ten opiera się o architekturę ARM Cortex-A8, bazuję na chipie firmy AllWinner sun4i.
\par
	\begin{table}[t]
		\centering
		\caption{SD Card Layout}
		\label{tab:sd-layout}
	\begin{tabular}{|c|c|c|}
	\hline
	\textbf{Start} & \textbf{rozmiar} & \textbf{użycie} \\ 
	\hline
	0 & 8KB & Nieużywane, dostępne do układu partycji \\
	\hline
	8 & 24KB & SPL \footnote{ang. emph{Secondary Program Loader}} \\
	\hline
	32 & 512KB & u-boot \\
	\hline
	544 & 128KB & Środowisko \\
	\hline
	672 & 352KB & Zarezerwowana \\
	\hline
	1024 & - & Dostępne dla partycji \\
	\hline
	
	
\end{tabular}
\end{table}

\par
Po włożniu karty SD do czytnika kart pamięci i podłączenia go do komputera, należy sprawdzić i upewnić się jak nazywa się podłączone urządzenie. Jest to o tyle ważne gdyż w systemach linux montowane jest w tym samym katalogu i zwykle z tym samym przedrostkiem co dysk twardy komputera. Gdyby karta pamięci zostałaby źle zidentyfikowana mogło być dojść do bezpowrotnego usunięcia danych z dysku twardego komputera. Po włożeniu karty do czytnika nazwę urządzenia w komputerze można sprawdzić za pomocą polecenia \lstinline{sudo dmesg} wyświetlając systemowy bufor komunikatów. 
\par
Pierwszym krokiem w przygotowaniu systemu jest skasowanie pierwszych 2047 sektorów pamięci na karcie pamięci - zgodnie z tym co widzimy na tabeli ~\ref{tab:sd-layout} pierwszy MB pamięci (2048 sektorów * 512B) jest zarezerwowany dla danych potrzebnych do uruchomienia systemu z karty.
\par
\begin{lstlisting}[language=bash]
dd if=/dev/zero of=/dev/sdh bs=512 count=2047
\end{lstlisting}
\par
Następnie za pomocą programu do partycjonowania (np. fdisk) należy utworzyć dwie partycje - pierwszą zaczynającą się w 2048 sektorze bądź dalszym o wielkości 16MB w której będzie przechowywane jądro (uImage), plik boot.scr i script.bin, oraz drugą partycja zajmująca pozostałą część karty, która będzie przechowywała główny system plików.
\par

\begin{lstlisting}[language=bash]
mkdir mnt
sudo mkfs.vfat /dev/sdh1
sudo mount /dev/sdh1 mnt
\end{lstlisting}

W powyższym fragmencie tworzony jest katalog mnt w domowym folderze projektu, pierwsza partycja na karcie jest formatowana do systemu plików FAT, później montowana jest do katalogu mnt.


Niezbędnymi narzędziami do zainstalowania systemu są:
\begin{itemize}
	\item {\lstinline{gcc-x-arm-linux-gnueabi}}  - kompilator dla architektury ARM. Fraza eabi, znajdująca się na końcu nazwy tego pakietu to pewien standard opracowany przez firmę ARM. ABI  \footnote{ang. \emph{Application Binary Interface}} jest to zestaw reguł i ustawień kompilacji, które decydują o tym jak dany program współpracuje z innymi bibliotekami, czy aplikacjami. Programy skompilowane przy pomocy kompilatora używającego standardu EABI są przenośne pomiędzy systemami operacyjnymi (jedyną barierą są biblioteki z którymi współpracuje dany program).
	\item {\lstinline{build-essential}} - jak sama nazwa wskazuje jest to zbiór pakietów niezbędnych do budowania oprogramowania. Zależnościami tego pakietu to m.in. g++, libc, czy make.
	\item {\lstinline{git}} - system kontroli wersji
	\item {\lstinline{u-boot-tools}} - zestaw narzędzi do bootloadera, który zostanie opisany w dalszej części pracy.
\end{itemize}


\subsubsection{Bootloader}

W komputerach o architekturze ARM proces uruchamiania systemu operacyjnego oparty jest o program uruchomieniowy Bootstrap. Na samym początku szuka on programu uruchomieniowego w pierwszym sektorze pamięci NAND, następnie przeszukiwana jest karta SD/MMC (szukając pliku boot.bin na pierwszej partycji karty), pamięć Dataflash, i jako ostatnia pamięć EEPROM podłączona do magistrali I2C. Jeżeli w którejś z tych lokacji bootstrap znajdzie prawidłowy kod, to jest on kopiowany do wewnętrznej pamięci operacyjnej procesora (SRAM) i tam uruchamiany. W następnej kolejności uruchamiany jest bootloader pierwszego poziomu, jest to zwykle mały program, którego zadaniem jest inicjalizacja pamięci RAM, oraz jednej z pamięci nieulotnych, a następnie uruchomienie bootloadera drugiego poziomu.
Najpopularniejszym bootloaderem w systemach opartych o architekturę ARM jest u-boot. Głównym zadaniem bootloadera drugiego poziomu jest wczytanie pliku jądra systemu. Za pomocą u-boota można go wczytać z karty SD, czy nawet poprzez serwer ftp.

\par

Aktualną wersję u-boota można pobrać i skompilować za pomocą poleceń:

\begin{lstlisting}[language=bash]
git clone https://github.com/linux-sunxi/u-boot-sunxi.git
cd uboot-allwinner
git checkout sun4i
make sun4i CROSS_COMPILE=arm-linux-gnueabi-
\end{lstlisting}

\par

W pierwszej lini pobierane jest repozytorium, w linii 3 z kolei przełączamy pobrane repozytorium na odpowiednią gałąź, wreszcie w linii 4 kompilujemy bootloader na architekturę \lstinline{sun4i} używając do tego celu kompilatora \lstinline{arm-linux-gnueabi-gcc}.

Po skompilowaniu można wgrać niezbędny pliki na kartę SD:

\begin{lstlisting}[language=bash]
dd if=spl/sun4i-spl.bin of=/dev/sdh bs=1024 seek=8  (if sun4i-spl.bin not found, try sunxi-spl.bin)
dd if=u-boot.bin of=/dev/sdh bs=1024 seek=32
\end{lstlisting}

\par

Plik boot.cmd jest plikiem konfiguracyjnym, używanym przez program bootstrap:

\begin{lstlisting}[language=bash]
setenv console 'ttyS0,115200'
setenv root '/dev/mmcblk0p2'
setenv panicarg 'panic=10'
setenv extra 'rootfstype=ext4 rootwait'
setenv loglevel '8'
setenv setargs 'setenv bootargs console=${console} root=${root} loglevel=${loglevel} ${panicarg} ${extra}'
setenv kernel 'uImage'
setenv boot_mmc 'fatload mmc 0 0x43000000 script.bin; fatload mmc 0 0x48000000 ${kernel}; bootm 0x48000000'
setenv bootcmd 'run setargs boot_mmc'
\end{lstlisting}

Stworzony plik boot.cmd należy skompilować do formatu boot.scr i skopiować na kartę SD:

\begin{lstlisting}[language=bash]
mkimage -A arm -O u-boot -T script -C none -n "boot" -d boot.cmd boot.scr
sudo cp boot.scr mnt/
\end{lstlisting}


\subsubsection{Kompilacja jądra i przygotowanie plików konfiguracyjnych}

Jądro systemu wykrywa i inicjuje sprzęt, oraz uruchamia proces init, który może być procesem dowolnego programu, natomiast aby uzyskać funkcjonalny system proces ten powinien zająć się uruchamianiem m.in. skryptów startowych, czy odbieraniem kodów wyjścia z zakończonych procesów.

Pierwszym krokiem jest ściągnięcie kodu źródłowego jądra, z racji tego, że architektura ARM Allwinner sun4i nie jest oficjalnie wspierana przez repozytorium w którym znajduję się jądro linuksa posłużono się repozytorium wywodzącym się z niego przystosowanym do potrzeb tej architektury.

Aktualną wersję jądra można pobrać za pomocą polecenia: \lstinline{git clone https://github.com/linux-sunxi/linux-sunxi}, a następnie przełączyć się na odpowiednią gałąź kodu: \lstinline{git checkout sunxi-3.4}. By skompilować jądro należy wykonać polecenia:

\begin{lstlisting}[language=bash]
make ARCH=arm CROSS_COMPILE=arm-linux-gnueabi- sun4i_defconfig
make ARCH=arm CROSS_COMPILE=arm-linux-gnueabi- -j16 uImage modules
make ARCH=arm CROSS_COMPILE=arm-linux-gnueabi- INSTALL_MOD_PATH=output modules_install

arm
sudo cp linux-allwinner/arch/arm/boot/uImage mnt/
\end{lstlisting}

W pierwszej linii generowana jest standardowa konfiguracja jądra dla danej platformy. Parametr \lstinline{ARCH=arm} określa architekturę jądra. Parametr \lstinline{CROSS_COMPILE=arm-linux-gnueabi-} określa przedrostek cross kompilatora, używany przy budowaniu i przy instalacji (czyli cross kompilator dla poleceń w liniach 1-3 to arm-linux-gnueabi-gcc).

\par

W linii drugiej budowane są moduły jądra, oraz plik z jądrem, następnie dodawana do niego jest suma kontrolna, oraz informacje wymagane przez bootloader. Parametr j16 jest parametrem programu \lstinline{make} określającym ile jednocześnie zadań może być uruchamianych, przez zadanie w przypadku programu \lstinline{make} rozumiane jest rozwiązanie jednej zależności. Program make za pomocą grafu zależności określa kolejność wykonywanych zadań.

\par
Komenda z linii trzeciej instaluje moduły w zadanej lokacji.
\par
Plik script.bin jest plikiem konfiguracyjnym dla danej platformy. Można go wygenerować samemu, dokumentacja znajduję się na oficjalnej stronie architektury sunxi - \url{http://linux-sunxi.org/Fex_Guide}. a następnie skompilować plik konfiguracyjny do formatu bin za pomocą narzędzia \lstinline{fex2bin}. W pliku tym konfigurowane są wartości typowe dla urządzenia tj. adres MAC komputera, czy maksymalna rozdzielczość.  W internecie istnieje wiele gotowych plików konfiguracyjnych dla komputera MK 802 II, na potrzeby niniejszej pracy konfiguracja została pobrana ze strony - \url{http://dl.miniand.com/gamboita/script-mk802ii-1080p60.7z}. 

Po wygenerowaniu pliku należy go skopiować na kartę pamięci do pierwszej partycji \lstinline{sudo cp script.bin mnt/script.bin}. 

Tym sposobem zakończono modyfikacje na partycji pierwszej - przygotowano i skopiowano plik jądra, plik konfiguracyjny urządzenia, oraz plik konfiguracyjny bootloadera. 

By odmontować partycje należy wpisać \lstinline{sudo umount mnt}

\subsubsection{Przygotowanie głównego systemu plików}

Pakiet debootstrap jest to narzędzie, który pozwala zainstalować bazowy system debiana w podkatalogu na obecnie zainstalowanym systemie. Jest on dostępny w oficjalnych repozytoriach debiana.

\begin{lstlisting}[language=bash]
sudo apt-get install debootstrap
mkdir debian-rootfs
cd debian-rootfs
dd if=/dev/zero of=debian_rootfs.img bs=1M count=1024
sudo mkfs.ext3 -F debian_rootfs.img
mkdir mnt
sudo mount -o loop debian_rootfs.img mnt
sudo debootstrap --verbose --arch armel --variant=minbase --foreign squeeze mnt http://ftp.debian.org/debian
\end{lstlisting}

W liniach 4-8 powyższego listingu tworzony jest pusty obraz systemu sformatowany do formatu ext3, oraz zamontowany do katalogu mnt. Za pomocą wspomnianego wcześniej narzędzia debootstrap ściągnięto z oficjalnej strony dystrybucji bazowy system operacyjny, oraz zainstalowano go.


\begin{lstlisting}[language=bash]
sudo apt-get install -t testing qemu-user-static
sudo apt-get install binfmt-support
sudo modprobe binfmt_misc
sudo cp /usr/bin/qemu-arm-static mnt/usr/bin
sudo mkdir mnt/dev/pts
sudo mount -t devpts devpts mnt/dev/pts
sudo mount -t proc proc mnt/proc
sudo chroot mnt/
/debootstrap/debootstrap --second-stage
\end{lstlisting}

Pakiet \lstinline{quemu-user-static} pozwala na emulowanie binarek zbudowanych statycznie. Ogólnie rzecz biorąc pozwala na uruchamianie procesów linuksowych na procesorze pod który nie zostały one skompilowane. Pakiet binfmt-support pozwala na obsługe dodatkowych formatów binarnych. Za pomocą polecenia \lstinline{modprobe} moduł obsługi dodatkowych formatów binarnych zostaje włączony jako moduł do jądra. Polecenie z linii 4 kopiuje aplikacje do nowo stworzonego systemu. Linie 5-6 to przygotowaniu interfejsu tzw. pseudoterminala. Poprzez zamontowanie systemu plików \lstinline{/proc} do nowo powstającego systemu, po wykonaniu chroota będzie możliwość korzystania z informacji dostarczonych przez jądro systemu. Wreszcie wykonujemy polecenie \lstinline{chroot} \footnote{ang. \emph{change root}}, które jak sama nazwa wskazuje zmienia główny system plików. Po wykonaniu przed ostatniego polecenia linia poleceń powinna wyświetlić komunikat \lstinline{I have no name!@hostname:/#}, ostatnie polecenie uruchamia wszystkie skrypty konfiguracyjne, które muszą zostać wykonane na architekturze docelowej (dlatego właśnie skopiowany został do /usr/bin program qemu-user-static). Mając uruchomiony system na komputerze wykonano podstawową konfiguracje dla systemu debian.
\\
Edycja repozytoriów menadżera pakietów:
\begin{lstlisting}[language=bash]
cd /root
cat <<koniec > /etc/apt/sources.list
deb http://security.debian.org/ squeeze/updates main contrib non-free
deb-src http://security.debian.org/ squeeze/updates main contrib non-free
deb http://ftp.debian.org/debian/  squeeze main contrib non-free
deb-src http://ftp.debian.org/debian/  squeeze main contrib non-free
koniec
apt-get update
\end{lstlisting}

Konfiguracja języka i podstawowych narzędzi:
\begin{lstlisting}[language=bash]
export LANG=C
apt-get install apt-utils
apt-get install dialog
apt-get install locales
cat <<END > /etc/apt/apt.conf.d/71mele
APT::Install-Recommends "0";
APT::Install-Suggests "0";
END
dpkg-reconfigure locales
export LANG=pl_PL.UTF-8
apt-get install dhcp3-client udev netbase ifupdown iproute openssh-server \
    iputils-ping wget net-tools ntpdate uboot-mkimage uboot-envtools vim nano less X
\end{lstlisting}

Po skonfigurowaniu usług internetowych, i konta roota, należy wyjść z trybu \lstinline{chroot}, oraz odmontować partycję:

\begin{lstlisting}[language=bash]
exit
sudo umount mnt/proc
sudo umount mnt/dev/pts
\end{lstlisting}

Ostatnim elementem w przygotowywaniu drugiej partycji jest zainstalowaniu modułów jądra na głównym systemie plików, poleceniem: \lstinline{sudo make ARCH=arm CROSS_COMPILE=arm-linux-gnueabi- INSTALL_MOD_PATH=../debian-rootfs/mnt modules_install}.

Po zainstalowaniu modułów pozostaje tylko przygotować drugą partycje na karcie SD i zainstalować przygotowany system plików:
\begin{lstlisting}[language=bash]
cd /home/share/mele-hacking
mkdir mnt2
sudo mkfs.ext4 /dev/sdh2 
sudo mount /dev/sdh2 mnt2
sudo cp -a debian-rootfs/mnt/* mnt2/
sudo umount mnt2
sudo umount debian-rootfs/mnt
\end{lstlisting}

\subsection{Uruchomienie oprogramowania generujące zdarzenia urządzeń wejścia dla systemów Mac OS X i Linux}


Aby uruchomić oprogramowanie klienta Linux należy pobrać repozytorium dostępne pod adresem: \url{https://github.com/athlan/remotely}, a następnie w folderze client wybrać dostępny system (macosx lub linux), i w odpowiednim podfolderze skompilować oprogramowanie poleceniem \lstinline{make}.
Po pomyślnej kompilacji należy uruchomić serwer poleceniem \lstinline{./bin/server}.


\subsection{Uruchomienie serwera www zarządzającego infrastrukturą}

W celu zainstalowania serwera www zarządzającego infrastrukturą należy zainstalować framework play, a następnie pobrać z repozytorium kod źródłowy aplikacji zarządzającej systemem dostępy pod adresem - \url{https://github.com/badeleux/remotely-web}. Uruchomienie serwera za pomocą polecenia \lstinline|play ~run|, rozwiąże wszystkie zależności, oraz skompiluje kod. 

\subsection{Instalacja i uruchamianie Varnish cache}

Aby poprawnie skonfigurować Varnish cache należy zainstalowany serwer Apache\ref{app:apache} ustawić na nasłuchiwanie portu innego, niż 80 (przychodzące połączenia dla protokołu HTTP), bowiem serwer Varnish będzie przyjmował połączenia przechodzące.

W pliku \lstinline{/etc/apache2/ports.conf} oraz \lstinline{/etc/apache2/sites-available/default} należy ustawić port na 81:
\begin{lstlisting}[language=bash]
NameVirtualHost 127.0.0.1:81
Listen 127.0.0.1:81
\end{lstlisting}

\begin{lstlisting}[language=bash]
<VirtualHost 127.0.0.1:81>
\end{lstlisting}

Należy przejść do instalacji Varnish cache, do pliku z repozytoriami (\lstinline{/etc/apt/sources.list}) należy dodać:
\begin{lstlisting}[language=bash]
deb http://repo.varnish-cache.org/ubuntu/ lucid varnish-3.0
\end{lstlisting}

Instalacja jest kontynuowana poprzez wykonanie:
\begin{lstlisting}[language=bash]
curl http://repo.varnish-cache.org/debian/GPG-key.txt | sudo apt-key add -
sudo apt-get update
sudo apt-get install varnish
\end{lstlisting}

Varnish został zainstalowany, należy go skonfigurować w pliku \lstinline{/etc/default/varnish}:

\begin{lstlisting}[language=bash]
DAEMON_OPTS="-a :80 \
            -T localhost:6082 \
            -f /etc/varnish/default.vcl \
            -S /etc/varnish/secret \
            -s malloc,256m"
\end{lstlisting}

Następnie w pliku \lstinline{/etc/varnish/default.vcl} należy zdefiniować, na jaki serwer przekierowywane są żądania o treść. Należy wskazać serwer Apache nasłuchujący na porcie 81:

\begin{lstlisting}[language=bash]
backend default {
    .host = "127.0.0.1";
    .port = "8080";
}
\end{lstlisting}

Obie usługi wymagają zrestartowania:

\begin{lstlisting}[language=bash]
service apache2 restart
service varnish restart
\end{lstlisting}
